% Options for packages loaded elsewhere
\PassOptionsToPackage{unicode}{hyperref}
\PassOptionsToPackage{hyphens}{url}
%
\documentclass[
]{article}
\usepackage{amsmath,amssymb}
\usepackage{iftex}
\ifPDFTeX
  \usepackage[T1]{fontenc}
  \usepackage[utf8]{inputenc}
  \usepackage{textcomp} % provide euro and other symbols
\else % if luatex or xetex
  \usepackage{unicode-math} % this also loads fontspec
  \defaultfontfeatures{Scale=MatchLowercase}
  \defaultfontfeatures[\rmfamily]{Ligatures=TeX,Scale=1}
\fi
\usepackage{lmodern}
\ifPDFTeX\else
  % xetex/luatex font selection
\fi
% Use upquote if available, for straight quotes in verbatim environments
\IfFileExists{upquote.sty}{\usepackage{upquote}}{}
\IfFileExists{microtype.sty}{% use microtype if available
  \usepackage[]{microtype}
  \UseMicrotypeSet[protrusion]{basicmath} % disable protrusion for tt fonts
}{}
\makeatletter
\@ifundefined{KOMAClassName}{% if non-KOMA class
  \IfFileExists{parskip.sty}{%
    \usepackage{parskip}
  }{% else
    \setlength{\parindent}{0pt}
    \setlength{\parskip}{6pt plus 2pt minus 1pt}}
}{% if KOMA class
  \KOMAoptions{parskip=half}}
\makeatother
\usepackage{xcolor}
\usepackage[margin=1in]{geometry}
\usepackage{color}
\usepackage{fancyvrb}
\newcommand{\VerbBar}{|}
\newcommand{\VERB}{\Verb[commandchars=\\\{\}]}
\DefineVerbatimEnvironment{Highlighting}{Verbatim}{commandchars=\\\{\}}
% Add ',fontsize=\small' for more characters per line
\usepackage{framed}
\definecolor{shadecolor}{RGB}{248,248,248}
\newenvironment{Shaded}{\begin{snugshade}}{\end{snugshade}}
\newcommand{\AlertTok}[1]{\textcolor[rgb]{0.94,0.16,0.16}{#1}}
\newcommand{\AnnotationTok}[1]{\textcolor[rgb]{0.56,0.35,0.01}{\textbf{\textit{#1}}}}
\newcommand{\AttributeTok}[1]{\textcolor[rgb]{0.13,0.29,0.53}{#1}}
\newcommand{\BaseNTok}[1]{\textcolor[rgb]{0.00,0.00,0.81}{#1}}
\newcommand{\BuiltInTok}[1]{#1}
\newcommand{\CharTok}[1]{\textcolor[rgb]{0.31,0.60,0.02}{#1}}
\newcommand{\CommentTok}[1]{\textcolor[rgb]{0.56,0.35,0.01}{\textit{#1}}}
\newcommand{\CommentVarTok}[1]{\textcolor[rgb]{0.56,0.35,0.01}{\textbf{\textit{#1}}}}
\newcommand{\ConstantTok}[1]{\textcolor[rgb]{0.56,0.35,0.01}{#1}}
\newcommand{\ControlFlowTok}[1]{\textcolor[rgb]{0.13,0.29,0.53}{\textbf{#1}}}
\newcommand{\DataTypeTok}[1]{\textcolor[rgb]{0.13,0.29,0.53}{#1}}
\newcommand{\DecValTok}[1]{\textcolor[rgb]{0.00,0.00,0.81}{#1}}
\newcommand{\DocumentationTok}[1]{\textcolor[rgb]{0.56,0.35,0.01}{\textbf{\textit{#1}}}}
\newcommand{\ErrorTok}[1]{\textcolor[rgb]{0.64,0.00,0.00}{\textbf{#1}}}
\newcommand{\ExtensionTok}[1]{#1}
\newcommand{\FloatTok}[1]{\textcolor[rgb]{0.00,0.00,0.81}{#1}}
\newcommand{\FunctionTok}[1]{\textcolor[rgb]{0.13,0.29,0.53}{\textbf{#1}}}
\newcommand{\ImportTok}[1]{#1}
\newcommand{\InformationTok}[1]{\textcolor[rgb]{0.56,0.35,0.01}{\textbf{\textit{#1}}}}
\newcommand{\KeywordTok}[1]{\textcolor[rgb]{0.13,0.29,0.53}{\textbf{#1}}}
\newcommand{\NormalTok}[1]{#1}
\newcommand{\OperatorTok}[1]{\textcolor[rgb]{0.81,0.36,0.00}{\textbf{#1}}}
\newcommand{\OtherTok}[1]{\textcolor[rgb]{0.56,0.35,0.01}{#1}}
\newcommand{\PreprocessorTok}[1]{\textcolor[rgb]{0.56,0.35,0.01}{\textit{#1}}}
\newcommand{\RegionMarkerTok}[1]{#1}
\newcommand{\SpecialCharTok}[1]{\textcolor[rgb]{0.81,0.36,0.00}{\textbf{#1}}}
\newcommand{\SpecialStringTok}[1]{\textcolor[rgb]{0.31,0.60,0.02}{#1}}
\newcommand{\StringTok}[1]{\textcolor[rgb]{0.31,0.60,0.02}{#1}}
\newcommand{\VariableTok}[1]{\textcolor[rgb]{0.00,0.00,0.00}{#1}}
\newcommand{\VerbatimStringTok}[1]{\textcolor[rgb]{0.31,0.60,0.02}{#1}}
\newcommand{\WarningTok}[1]{\textcolor[rgb]{0.56,0.35,0.01}{\textbf{\textit{#1}}}}
\usepackage{graphicx}
\makeatletter
\def\maxwidth{\ifdim\Gin@nat@width>\linewidth\linewidth\else\Gin@nat@width\fi}
\def\maxheight{\ifdim\Gin@nat@height>\textheight\textheight\else\Gin@nat@height\fi}
\makeatother
% Scale images if necessary, so that they will not overflow the page
% margins by default, and it is still possible to overwrite the defaults
% using explicit options in \includegraphics[width, height, ...]{}
\setkeys{Gin}{width=\maxwidth,height=\maxheight,keepaspectratio}
% Set default figure placement to htbp
\makeatletter
\def\fps@figure{htbp}
\makeatother
\setlength{\emergencystretch}{3em} % prevent overfull lines
\providecommand{\tightlist}{%
  \setlength{\itemsep}{0pt}\setlength{\parskip}{0pt}}
\setcounter{secnumdepth}{-\maxdimen} % remove section numbering
\ifLuaTeX
  \usepackage{selnolig}  % disable illegal ligatures
\fi
\usepackage{bookmark}
\IfFileExists{xurl.sty}{\usepackage{xurl}}{} % add URL line breaks if available
\urlstyle{same}
\hypersetup{
  pdftitle={MODUL LOGIKA DAN PENGULANGAN},
  pdfauthor={Zahra Mahendra Putri, Regina Dwirahma Alisya},
  hidelinks,
  pdfcreator={LaTeX via pandoc}}

\title{MODUL LOGIKA DAN PENGULANGAN}
\author{Zahra Mahendra Putri, Regina Dwirahma Alisya}
\date{2024-12-06}

\begin{document}
\maketitle

\section{BAB 1 : DASAR PEMOGRAMAN R}\label{bab-1-dasar-pemograman-r}

\subsection{1.1 Pengertian}\label{pengertian}

\begin{enumerate}
\def\labelenumi{\arabic{enumi}.}
\item
  Statement If

  Fungsi if adalah statement yang umum digunakan pada percabangan. If
  adalah fungsi untuk membuat perbandingan logis antara nilai dan apa
  yang diharapkan dengan menguji kondisi dan mengembalikan hasil jika
  True atau False.
\item
  If Else

  Fungsi ini di gunakan jika ada perintah yang akan dijalankan ketika
  masing-masing kondisi bernilai True atau False. Fungsi ini juga
  dikenal sebagai percabangan bersyarat, Ifelse berfungsi untuk
  memastikan bahwa instruksi tertentu hanya akan dijalankan jika suatu
  kondisi terpenuhi.
\item
  For loop

  For loop merupakan jenis loop yang paling sederhana dan sering
  digunakan. For-loop digunakan untuk mengulang sekumpulan objek,
  seperti vektor, daftar, matriks, atau kerangka data, dan menerapkan
  serangkaian operasi yang sama pada setiap item dari struktur data
  tertentu. Selain for loop, ada juga jenis loop lainnya, yaitu while
  loop dan repeat loop.
\item
  While loop

  While loop merupakan loop yang digunakan ketika kita telah menetapkan
  stop condition sebelumnya. Blok statement/kode yang sama akan terus
  dijalankan sampai stop condition ini tercapai. Stop condition akan di
  cek sebelum melakukan proses loop.
\item
  Repeat loop

  Fungsi repeat loop adalah~untuk menjalankan kode atau statement yang
  sama secara berulang-ulang hingga kondisi yang ditentukan tercapai.
\item
  Nested Ifelse

  Fungsi nested if else adalah~untuk memeriksa kondisi baru setelah
  kondisi sebelumnya telah ditemukan benar atau salah.~Nested if else
  juga dikenal sebagai if bersarang, yaitu kondisi yang di dalamnya
  terdapat kondisi lagi
\item
  Fungsi logika ``and'', ``or''

  Fungsi logika ``and'' dan ``or'' di RStudio berfungsi
  untuk~membandingkan dua kondisi logika, yaitu logika benar (TRUE) dan
  logika salah (FALSE). Operator ``and'' ini akan mengembalikan TRUE
  jika kedua nilai logika yang dibandingkan bernilai TRUE sedangkan
  operator ``or'' ini akan mengembalikan TRUE jika kedua nilai logika
  yang dibandingkan bernilai TRUE.
\end{enumerate}

\subsection{1.2 Pentingnya untuk analisis
data}\label{pentingnya-untuk-analisis-data}

\begin{enumerate}
\def\labelenumi{\arabic{enumi}.}
\item
  Statement If

  Fungsi if adalah statement yang umum digunakan pada percabangan. If
  adalah fungsi untuk membuat perbandingan logis antara nilai dan apa
  yang diharapkan dengan menguji kondisi dan mengembalikan hasil jika
  True atau False.
\item
  If Else

  Fungsi ini di gunakan jika ada perintah yang akan dijalankan ketika
  masing-masing kondisi bernilai True atau False. Fungsi ini juga
  dikenal sebagai percabangan bersyarat, Ifelse berfungsi untuk
  memastikan bahwa instruksi tertentu hanya akan dijalankan jika suatu
  kondisi terpenuhi.
\item
  For loop

  For loop merupakan jenis loop yang paling sederhana dan sering
  digunakan. For-loop digunakan untuk mengulang sekumpulan objek,
  seperti vektor, daftar, matriks, atau kerangka data, dan menerapkan
  serangkaian operasi yang sama pada setiap item dari struktur data
  tertentu. Selain for loop, ada juga jenis loop lainnya, yaitu while
  loop dan repeat loop.
\item
  While loop

  While loop digunakan untuk mengulang proses selama kondisi tertentu
  terpenuhi. Singkatnya, misalkan kita ingin menabung 10 juta, tetapi
  banyaknya waktu yang dibutuhkan untuk mencapai target Tabungan
  tersebut tidak diketahui, namun di sisi lain kita mengetahui kapan
  harus berhenti menabung disaat target sudah tercapai, bagaimana kita
  mengetahuinya? Maka kita dapat mengetahuinya ketika tabungan itu
  mencapai target.
\item
  Repeat Loop

  Repeat loop dalam analisis data digunakan untuk menjalankan perintah
  tertentu secara berulang hingga kondisi syarat tersebut
  telah~terpenuhi
\item
  Nested Ifelse

  Fungsi nested if else adalah~untuk memeriksa kondisi baru setelah
  kondisi sebelumnya telah ditemukan benar atau salah.~Nested if else
  juga dikenal sebagai if bersarang, yaitu kondisi yang di dalamnya
  terdapat kondisi lagi
\item
  Fungsi logika ``and'', ``or''

  Fungsi logika ``and'' dan ``or'' di RStudio berfungsi
  untuk~membandingkan dua kondisi logika, yaitu logika benar (TRUE) dan
  logika salah (FALSE). Operator ``and'' ini akan mengembalikan TRUE
  jika kedua nilai logika yang dibandingkan bernilai TRUE sedangkan
  operator ``or'' ini akan mengembalikan TRUE jika kedua nilai logika
  yang dibandingkan bernilai TRUE.
\end{enumerate}

\section{BAB 2 : LOGIKA DAN PENGULANGAN DALAM
R}\label{bab-2-logika-dan-pengulangan-dalam-r}

\subsection{2.1 Function if}\label{function-if}

Bentuk umum dari perintah if() adalah if(kondisi) ekspresi dan argumen
yang dibutuhkan pada fungsi if() adalah sebuah vector logical tunggal
bernilai TRUE atau FALSE. Berikut ini adalah sintaks dari fungsi If.

if(kondisi) \{

\ldots{}

\}

\subsection{\texorpdfstring{2.2 \textbf{Function
ifelse}}{2.2 Function ifelse}}\label{function-ifelse}

Bentuk umum dari perintah if() adalah if(kondisi) ekspresi dan argumen
yang dibutuhkan pada fungsi if() adalah sebuah vector logical tunggal
bernilai TRUE atau FALSE. Berikut ini adalah sintaks dari fungsi If.

if(kondisi) \{

\ldots{}

\}

\subsection{2.3 Function for loop}\label{function-for-loop}

Pengulangan menggunakan statement for, Berikut ini sintaks dari for
loop.

for(i in 1:n) \{

print(i)

\}

\subsection{2.4 Funciton While Loop}\label{funciton-while-loop}

Cara lain untuk melakukan pengulangan adalah dengan menggunakan
statement while. Sintaks dari statement ini adalah sebagai berikut.

i = 1;

while(i \textless= n)\{

print(i);

i = i+1;

\}

\subsection{2.5 Repeat Loop}\label{repeat-loop}

Repeat digunakan untuk menjalankan blok pernyataan berulang kali hingga
pernyataan break jump ditemukan dan untuk mengakhiri atau keluar dari
loop harus menggunakan pernyataan break. Jika tidak menggunakan ini,
pernyataan repeat akan berakhir dalam loop tak terbatas. Berikut ini
adalah sintaks dari repeat loop.

i = 1

repeat\{

print(i)

if(i==n) \{break()\}

i = i + 1

\}

\subsection{2.6 Function Nested ifelse}\label{function-nested-ifelse}

Nested Ifelse adalah fungsi If yang berada di dalam If lainnya. Semakin
menjorok ke dalam, maka pernyataan tersebut akan semakin ada di Tingkat
bawah. Berikut ini sintaks dari nested ifelse.

If (kondisi 1) \{

If (kondisi 2) \{

\} else \{

\}

\subsection{2.7 Function ``and (\&)'',
``or(\textbar)''}\label{function-and-or}

\begin{enumerate}
\def\labelenumi{\arabic{enumi}.}
\tightlist
\item
  Fungsi ``and''
\end{enumerate}

\begin{verbatim}
Operator “and” diwakili dengan tanda ampersand ‘&’ dan ‘&&’. Berikut ini adalah sintaks dari logika “and”.
\end{verbatim}

If \{kondisi 1\&kondisi 2) \{

\} else \{

\}

\begin{enumerate}
\def\labelenumi{\arabic{enumi}.}
\setcounter{enumi}{1}
\tightlist
\item
  Fungsi ``or''
\end{enumerate}

\begin{verbatim}
Operator “or” diwakili dengan symbol pike ( \| ) dan ( \|\| ). Berikut ini adalah sintaks dari logika “or”.
\end{verbatim}

If (kondisi 1 \textbar\textbar{} kondisi~2)~\{

\}~else~\{

\}

\section{BAB 3 : Implementasi study kasus
nyata}\label{bab-3-implementasi-study-kasus-nyata}

\subsection{3.1 STUDY KASUS IF}\label{study-kasus-if}

Sebuah perusahaan ingin memberikan bonus 10\% dari gaji berdasarkan
performa kerjanya. karyawan layak mendapat bonus jika peformanya diatas
80. Peforma yang dimiliki oleh karyawan yaitu 85 dengan gaji 6..000.000.
Apakah seorang karyawan berhak mendapatkan bonus ?

\begin{Shaded}
\begin{Highlighting}[]
\NormalTok{gaji }\OtherTok{\textless{}{-}} \DecValTok{6000000}
\NormalTok{peforma }\OtherTok{\textless{}{-}} \DecValTok{85}
\CommentTok{\#fungsi if}
\ControlFlowTok{if}\NormalTok{ (peforma }\SpecialCharTok{\textgreater{}} \DecValTok{80}\NormalTok{) \{}
\NormalTok{  bonus }\OtherTok{\textless{}{-}}\NormalTok{ gaji }\SpecialCharTok{*} \FloatTok{0.1}
  \FunctionTok{print}\NormalTok{(}\FunctionTok{paste}\NormalTok{(}\StringTok{"Karyawan mendapat bonus sebesar :"}\NormalTok{, bonus))}
\NormalTok{\} }\ControlFlowTok{else}\NormalTok{ \{}
  \FunctionTok{print}\NormalTok{(}\FunctionTok{paste}\NormalTok{(}\StringTok{"Karyawan tidak mendapat bonus sebesar."}\NormalTok{))}
\NormalTok{\}}
\end{Highlighting}
\end{Shaded}

\begin{verbatim}
## [1] "Karyawan mendapat bonus sebesar : 6e+05"
\end{verbatim}

\subsection{3.2 STUDY KASUS IFELSE}\label{study-kasus-ifelse}

Sebuah perusahaan memiliki data gaji dan peforma kerja beberapa
karyawan. karyawan mendapatkan bonus 10\% dari gaji jika peforma mereka
diatas 85. tampilkan data karyawan, peforma, besaran bonus dan status
apakah beberapa karyawan di perusahaan tersebut layak mendapatkan bonus.

\begin{Shaded}
\begin{Highlighting}[]
\NormalTok{gaji }\OtherTok{\textless{}{-}} \FunctionTok{c}\NormalTok{(}\DecValTok{6000000}\NormalTok{, }\DecValTok{5000000}\NormalTok{, }\DecValTok{4500000}\NormalTok{, }\DecValTok{7000000}\NormalTok{, }\DecValTok{8000000}\NormalTok{, }\DecValTok{3000000}\NormalTok{)}
\NormalTok{peforma }\OtherTok{\textless{}{-}} \FunctionTok{c}\NormalTok{(}\DecValTok{86}\NormalTok{, }\DecValTok{70}\NormalTok{, }\DecValTok{75}\NormalTok{, }\DecValTok{80}\NormalTok{, }\DecValTok{90}\NormalTok{, }\DecValTok{75}\NormalTok{)}
\CommentTok{\#menghitung bonus }
\NormalTok{bonus }\OtherTok{\textless{}{-}} \FunctionTok{ifelse}\NormalTok{(peforma }\SpecialCharTok{\textgreater{}} \DecValTok{85}\NormalTok{, gaji }\SpecialCharTok{*}\FloatTok{0.1}\NormalTok{, }\DecValTok{0}\NormalTok{)}
\NormalTok{keterangan }\OtherTok{\textless{}{-}} \FunctionTok{ifelse}\NormalTok{(peforma }\SpecialCharTok{\textgreater{}} \DecValTok{85}\NormalTok{, }\StringTok{"Karyawan mendapatkan bonus:"}\NormalTok{, }\StringTok{"Karyawan tidak mendapatkan bonus"}\NormalTok{)}
\NormalTok{data\_karyawan }\OtherTok{\textless{}{-}} \FunctionTok{data.frame}\NormalTok{(}\AttributeTok{Gaji =}\NormalTok{ gaji, }\AttributeTok{performa =}\NormalTok{ peforma, }\AttributeTok{Bonus =}\NormalTok{ bonus, }\AttributeTok{Status =}\NormalTok{ keterangan)}
\FunctionTok{print}\NormalTok{(data\_karyawan)}
\end{Highlighting}
\end{Shaded}

\begin{verbatim}
##      Gaji performa Bonus                           Status
## 1 6000000       86 6e+05      Karyawan mendapatkan bonus:
## 2 5000000       70 0e+00 Karyawan tidak mendapatkan bonus
## 3 4500000       75 0e+00 Karyawan tidak mendapatkan bonus
## 4 7000000       80 0e+00 Karyawan tidak mendapatkan bonus
## 5 8000000       90 8e+05      Karyawan mendapatkan bonus:
## 6 3000000       75 0e+00 Karyawan tidak mendapatkan bonus
\end{verbatim}

\subsection{3.3 For loop}\label{for-loop}

pengecekan angka 1-30 menggunakan for loop untuk menentukan ganjil dan
genap dari sebuah bilangan menggunakan for loop

\begin{Shaded}
\begin{Highlighting}[]
\ControlFlowTok{for}\NormalTok{ (i }\ControlFlowTok{in} \DecValTok{1}\SpecialCharTok{:}\DecValTok{30}\NormalTok{) \{          }\CommentTok{\#for loop untuk mengecek angka ganjil atau genap}
  \ControlFlowTok{if}\NormalTok{ (i }\SpecialCharTok{\%\%} \DecValTok{2} \SpecialCharTok{==} \DecValTok{0}\NormalTok{) \{       }\CommentTok{\#jika angka dibagi 2 bersisa 0 maka itu genap}
    \FunctionTok{cat}\NormalTok{(i, }\StringTok{"adalah angka genap}\SpecialCharTok{\textbackslash{}n}\StringTok{"}\NormalTok{)}
\NormalTok{  \} }\ControlFlowTok{else}\NormalTok{ \{}
    \FunctionTok{cat}\NormalTok{(i, }\StringTok{"adalah angka ganjil}\SpecialCharTok{\textbackslash{}n}\StringTok{"}\NormalTok{)}
\NormalTok{  \}}
\NormalTok{\}}
\end{Highlighting}
\end{Shaded}

\begin{verbatim}
## 1 adalah angka ganjil
## 2 adalah angka genap
## 3 adalah angka ganjil
## 4 adalah angka genap
## 5 adalah angka ganjil
## 6 adalah angka genap
## 7 adalah angka ganjil
## 8 adalah angka genap
## 9 adalah angka ganjil
## 10 adalah angka genap
## 11 adalah angka ganjil
## 12 adalah angka genap
## 13 adalah angka ganjil
## 14 adalah angka genap
## 15 adalah angka ganjil
## 16 adalah angka genap
## 17 adalah angka ganjil
## 18 adalah angka genap
## 19 adalah angka ganjil
## 20 adalah angka genap
## 21 adalah angka ganjil
## 22 adalah angka genap
## 23 adalah angka ganjil
## 24 adalah angka genap
## 25 adalah angka ganjil
## 26 adalah angka genap
## 27 adalah angka ganjil
## 28 adalah angka genap
## 29 adalah angka ganjil
## 30 adalah angka genap
\end{verbatim}

\subsection{3.4 While loop}\label{while-loop}

pengecekan angka 1-30 menggunakan for loop untuk menentukan ganjil dan
genap dari sebuah bilangan menggunakan for loop

\begin{Shaded}
\begin{Highlighting}[]
\NormalTok{i }\OtherTok{\textless{}{-}} \DecValTok{1} \CommentTok{\#mulai dari angka 1}
\ControlFlowTok{while}\NormalTok{ (i }\SpecialCharTok{\textless{}=} \DecValTok{30}\NormalTok{) \{}
  \ControlFlowTok{if}\NormalTok{ (i }\SpecialCharTok{\%\%} \DecValTok{2} \SpecialCharTok{==} \DecValTok{0}\NormalTok{) \{       }\CommentTok{\#jika angka dibagi 2 bersisa 0 maka itu genap}
    \FunctionTok{cat}\NormalTok{(i, }\StringTok{"adalah angka genap}\SpecialCharTok{\textbackslash{}n}\StringTok{"}\NormalTok{)}
\NormalTok{  \} }\ControlFlowTok{else}\NormalTok{ \{                 }\CommentTok{\#jika tidak berarti ganjil}
  \FunctionTok{cat}\NormalTok{(i, }\StringTok{"adalah angka ganjil}\SpecialCharTok{\textbackslash{}n}\StringTok{"}\NormalTok{)}
\NormalTok{  \}}
\NormalTok{  i }\OtherTok{\textless{}{-}}\NormalTok{ i }\SpecialCharTok{+} \DecValTok{1}    \CommentTok{\#setiap iterasi, i+1 untuk melanjutkan ke angka berikutnya}
\NormalTok{\}}
\end{Highlighting}
\end{Shaded}

\begin{verbatim}
## 1 adalah angka ganjil
## 2 adalah angka genap
## 3 adalah angka ganjil
## 4 adalah angka genap
## 5 adalah angka ganjil
## 6 adalah angka genap
## 7 adalah angka ganjil
## 8 adalah angka genap
## 9 adalah angka ganjil
## 10 adalah angka genap
## 11 adalah angka ganjil
## 12 adalah angka genap
## 13 adalah angka ganjil
## 14 adalah angka genap
## 15 adalah angka ganjil
## 16 adalah angka genap
## 17 adalah angka ganjil
## 18 adalah angka genap
## 19 adalah angka ganjil
## 20 adalah angka genap
## 21 adalah angka ganjil
## 22 adalah angka genap
## 23 adalah angka ganjil
## 24 adalah angka genap
## 25 adalah angka ganjil
## 26 adalah angka genap
## 27 adalah angka ganjil
## 28 adalah angka genap
## 29 adalah angka ganjil
## 30 adalah angka genap
\end{verbatim}

\subsection{3.5 Repeat Loop}\label{repeat-loop-1}

Membuat program menggunakan repeat loop yang akan menambahkan angka dari
1,2,3 dan seterusnya hingga jumlah atau totalnya mencapai 50. ketika
total pencapaian melebihi 50 maka program akan berhenti.

\begin{Shaded}
\begin{Highlighting}[]
\NormalTok{total }\OtherTok{\textless{}{-}} \DecValTok{0}
\NormalTok{awal }\OtherTok{\textless{}{-}} \DecValTok{1}
\ControlFlowTok{repeat}\NormalTok{\{}
\NormalTok{  total }\OtherTok{\textless{}{-}}\NormalTok{ total }\SpecialCharTok{+}\NormalTok{ awal}
  \FunctionTok{cat}\NormalTok{(}\StringTok{"Menambahkan :"}\NormalTok{, awal, }\StringTok{"Total saat ini :"}\NormalTok{, total, }\StringTok{"}\SpecialCharTok{\textbackslash{}n}\StringTok{"}\NormalTok{)}
\NormalTok{  awal }\OtherTok{\textless{}{-}}\NormalTok{ awal}\SpecialCharTok{+}\DecValTok{1}
  \ControlFlowTok{if}\NormalTok{ (total }\SpecialCharTok{\textgreater{}=} \DecValTok{50}\NormalTok{) \{}
    \FunctionTok{cat}\NormalTok{(}\StringTok{"Penjumlahan selesai. Total akhir :"}\NormalTok{, total, }\StringTok{"}\SpecialCharTok{\textbackslash{}n}\StringTok{"}\NormalTok{)}
    \ControlFlowTok{break}
\NormalTok{  \}}
\NormalTok{\}}
\end{Highlighting}
\end{Shaded}

\begin{verbatim}
## Menambahkan : 1 Total saat ini : 1 
## Menambahkan : 2 Total saat ini : 3 
## Menambahkan : 3 Total saat ini : 6 
## Menambahkan : 4 Total saat ini : 10 
## Menambahkan : 5 Total saat ini : 15 
## Menambahkan : 6 Total saat ini : 21 
## Menambahkan : 7 Total saat ini : 28 
## Menambahkan : 8 Total saat ini : 36 
## Menambahkan : 9 Total saat ini : 45 
## Menambahkan : 10 Total saat ini : 55 
## Penjumlahan selesai. Total akhir : 55
\end{verbatim}

\subsection{3.6 Nested if-else}\label{nested-if-else}

penentuan nilai grade siswa berdasarkan nilai ujian :

Terdapat data nilai 6 siswa berikut :

\begin{enumerate}
\def\labelenumi{\arabic{enumi}.}
\tightlist
\item
  Patricia : 90
\item
  Faiz : 85
\item
  Kinan : 70
\item
  Andika : 60
\item
  Rafi : 75
\item
  Stella : 88
\end{enumerate}

\begin{itemize}
\item
  ``A : Sangat Baik'' jika nilai \textgreater= 85
\item
  ``B : Baik'' jika nilai 70-84
\item
  ``C : Cukup'' jika nilai 50-69
\item
  '' D : Kurang'' jika nilai \textless{} 50
\end{itemize}

\begin{Shaded}
\begin{Highlighting}[]
\NormalTok{nama }\OtherTok{\textless{}{-}} \FunctionTok{c}\NormalTok{ (}\StringTok{"Patricia"}\NormalTok{, }\StringTok{"Faiz"}\NormalTok{, }\StringTok{"Kinan"}\NormalTok{, }\StringTok{"Andika"}\NormalTok{, }\StringTok{"Rafi"}\NormalTok{, }\StringTok{"Stella"}\NormalTok{, }\StringTok{"Fabian"}\NormalTok{)}
\NormalTok{nilai }\OtherTok{\textless{}{-}} \FunctionTok{c}\NormalTok{(}\DecValTok{90}\NormalTok{, }\DecValTok{85}\NormalTok{, }\DecValTok{70}\NormalTok{, }\DecValTok{60}\NormalTok{, }\DecValTok{75}\NormalTok{, }\DecValTok{88}\NormalTok{, }\DecValTok{45}\NormalTok{)}
\NormalTok{keterangan }\OtherTok{\textless{}{-}} \FunctionTok{character}\NormalTok{(}\FunctionTok{length}\NormalTok{(nilai))}


\CommentTok{\#nested if{-}else}
\ControlFlowTok{for}\NormalTok{ (i }\ControlFlowTok{in} \DecValTok{1}\SpecialCharTok{:}\FunctionTok{length}\NormalTok{(nilai)) \{}
  \ControlFlowTok{if}\NormalTok{ (nilai[i] }\SpecialCharTok{\textgreater{}=} \DecValTok{85}\NormalTok{)\{}
\NormalTok{    keterangan[i] }\OtherTok{\textless{}{-}} \StringTok{"A : Sangat Baik"}
\NormalTok{  \} }\ControlFlowTok{else} \ControlFlowTok{if}\NormalTok{ (nilai[i] }\SpecialCharTok{\textgreater{}=} \DecValTok{70}\NormalTok{) \{}
\NormalTok{    keterangan[i] }\OtherTok{\textless{}{-}} \StringTok{"B : Baik"}
\NormalTok{  \} }\ControlFlowTok{else} \ControlFlowTok{if}\NormalTok{ (nilai[i] }\SpecialCharTok{\textgreater{}=} \DecValTok{50}\NormalTok{) \{}
\NormalTok{    keterangan[i] }\OtherTok{\textless{}{-}} \StringTok{"C : Cukup Baik"}
\NormalTok{  \} }\ControlFlowTok{else}\NormalTok{ \{}
\NormalTok{    keterangan[i] }\OtherTok{\textless{}{-}} \StringTok{"D : Kurang Baik"}
\NormalTok{  \}}
\NormalTok{\}}
\CommentTok{\#membuat menjadi data frame }
\NormalTok{hasil }\OtherTok{\textless{}{-}} \FunctionTok{data.frame}\NormalTok{(}
  \AttributeTok{Nama =}\NormalTok{ nama,}
  \AttributeTok{Nilai =}\NormalTok{ nilai,}
  \AttributeTok{Keterangan =}\NormalTok{ keterangan}
\NormalTok{)}
\FunctionTok{print}\NormalTok{(hasil)}
\end{Highlighting}
\end{Shaded}

\begin{verbatim}
##       Nama Nilai      Keterangan
## 1 Patricia    90 A : Sangat Baik
## 2     Faiz    85 A : Sangat Baik
## 3    Kinan    70        B : Baik
## 4   Andika    60  C : Cukup Baik
## 5     Rafi    75        B : Baik
## 6   Stella    88 A : Sangat Baik
## 7   Fabian    45 D : Kurang Baik
\end{verbatim}

\subsection{3.7 Logika Dasar and (\&), or
(\textbar)}\label{logika-dasar-and-or}

Universitas Sultan Ageng Tirtayasa menentukan status kelulusan mahasiswa
berdasarkan dua kriteria berikut :

\begin{enumerate}
\def\labelenumi{\arabic{enumi}.}
\tightlist
\item
  Nilai rata-rata mahasiswa harus \textgreater= 75
\item
  Kehadiran mahasiswa harus \textgreater= 80\%
\end{enumerate}

jika kedua kriteria terpenuhi, mahasiswa dinyatakan lulus. jika kedua
kriteria tidak terpenuhi, mahasiswa dinyatakan tidak lulus.

tentukan status kelulusan untuk daftar mahasiswa berikut :

\begin{itemize}
\item
  Stella : 85, Kehadiran : 85\%
\item
  Rafly : 65, Kehadiran : 70\%
\item
  Faiz : 78, Kehadiran : 85\%
\item
  Pingkan : 90, Kehadiran : 80\%
\item
  Zilda : 70, Kehadiran : 60\%
\item
  Kinan : 73, Kehadiran : 80\%
\end{itemize}

Tampilkan tabel yang menunjukkan Nama, Nilai rata-rata, Kehadiran, dan
Status Kelulusan masing-masing mahasiswa.

\begin{Shaded}
\begin{Highlighting}[]
\NormalTok{nama }\OtherTok{\textless{}{-}} \FunctionTok{c}\NormalTok{(}\StringTok{"Stella"}\NormalTok{, }\StringTok{"Rafly"}\NormalTok{, }\StringTok{"Faiz"}\NormalTok{, }\StringTok{"Pingkan"}\NormalTok{, }\StringTok{"Zilda"}\NormalTok{, }\StringTok{"Kinan"}\NormalTok{)}
\NormalTok{nilai }\OtherTok{\textless{}{-}} \FunctionTok{c}\NormalTok{(}\DecValTok{85}\NormalTok{, }\DecValTok{65}\NormalTok{, }\DecValTok{78}\NormalTok{, }\DecValTok{90}\NormalTok{, }\DecValTok{70}\NormalTok{, }\DecValTok{73}\NormalTok{)}
\NormalTok{kehadiran }\OtherTok{\textless{}{-}} \FunctionTok{c}\NormalTok{(}\DecValTok{85}\NormalTok{, }\DecValTok{70}\NormalTok{, }\DecValTok{85}\NormalTok{, }\DecValTok{80}\NormalTok{, }\DecValTok{60}\NormalTok{, }\DecValTok{80}\NormalTok{)}
\NormalTok{kelulusan }\OtherTok{\textless{}{-}} \FunctionTok{ifelse}\NormalTok{(nilai }\SpecialCharTok{\textgreater{}=} \DecValTok{75} \SpecialCharTok{\&}\NormalTok{ kehadiran }\SpecialCharTok{\textgreater{}=} \DecValTok{80}\NormalTok{, }\StringTok{"Lulus"}\NormalTok{, }\StringTok{"Tidak Lulus"}\NormalTok{)}
\CommentTok{\#status kelulusan untuk setiap mahasiswa }
\NormalTok{status }\OtherTok{\textless{}{-}} \FunctionTok{c}\NormalTok{()}
\ControlFlowTok{for}\NormalTok{ (i }\ControlFlowTok{in} \DecValTok{1}\SpecialCharTok{:}\FunctionTok{length}\NormalTok{(nama)) \{}
  \ControlFlowTok{if}\NormalTok{ (nilai[i] }\SpecialCharTok{\textgreater{}=} \DecValTok{75} \SpecialCharTok{\&\&}\NormalTok{ kehadiran[i] }\SpecialCharTok{\textgreater{}=} \DecValTok{80}\NormalTok{) \{}
\NormalTok{    status[i] }\OtherTok{\textless{}{-}} \StringTok{"Lulus"}
\NormalTok{  \} }\ControlFlowTok{else}\NormalTok{ \{}
\NormalTok{    status[i] }\OtherTok{\textless{}{-}} \StringTok{"Tidak Lulus"}
\NormalTok{  \}}
\NormalTok{\}}

\CommentTok{\#menampilkan hasil}
\NormalTok{hasil }\OtherTok{\textless{}{-}} \FunctionTok{data.frame}\NormalTok{(}\AttributeTok{Nama=}\NormalTok{nama, }\AttributeTok{Nilai=}\NormalTok{nilai, }\AttributeTok{Kehadiran=}\NormalTok{kehadiran, }\AttributeTok{Status=}\NormalTok{status)}
\FunctionTok{print}\NormalTok{(hasil)}
\end{Highlighting}
\end{Shaded}

\begin{verbatim}
##      Nama Nilai Kehadiran      Status
## 1  Stella    85        85       Lulus
## 2   Rafly    65        70 Tidak Lulus
## 3    Faiz    78        85       Lulus
## 4 Pingkan    90        80       Lulus
## 5   Zilda    70        60 Tidak Lulus
## 6   Kinan    73        80 Tidak Lulus
\end{verbatim}

\subsection{3.8 Logika Dasar or (\textbar)}\label{logika-dasar-or}

Universitas Sultan Ageng Tirtayasa menentukan status kelulusan mahasiswa
berdasarkan dua kriteria berikut :

\begin{enumerate}
\def\labelenumi{\arabic{enumi}.}
\tightlist
\item
  Nilai rata-rata mahasiswa harus \textgreater= 75
\item
  Kehadiran mahasiswa harus \textgreater= 80\%
\end{enumerate}

jika salah satu kriteria terpenuhi, mahasiswa dinyatakan lulus. jika
salah satu kriteria tidak terpenuhi, mahasiswa dinyatakan tidak lulus.

tentukan status kelulusan untuk daftar mahasiswa berikut :

\begin{itemize}
\item
  Stella : 85, Kehadiran : 85\%
\item
  Rafly : 65, Kehadiran : 70\%
\item
  Faiz : 78, Kehadiran : 85\%
\item
  Pingkan : 90, Kehadiran : 80\%
\item
  Zilda : 70, Kehadiran : 60\%
\item
  Kinan : 80, Kehadiran : 90\%
\end{itemize}

Tampilkan tabel yang menunjukkan Nama, Nilai rata-rata, Kehadiran, dan
Status Kelulusan masing-masing mahasiswa.

\begin{Shaded}
\begin{Highlighting}[]
\NormalTok{nama }\OtherTok{\textless{}{-}} \FunctionTok{c}\NormalTok{(}\StringTok{"Stella"}\NormalTok{, }\StringTok{"Rafly"}\NormalTok{, }\StringTok{"Faiz"}\NormalTok{, }\StringTok{"Pingkan"}\NormalTok{, }\StringTok{"Zilda"}\NormalTok{, }\StringTok{"Kinan"}\NormalTok{)}
\NormalTok{nilai }\OtherTok{\textless{}{-}} \FunctionTok{c}\NormalTok{(}\DecValTok{85}\NormalTok{, }\DecValTok{65}\NormalTok{, }\DecValTok{78}\NormalTok{, }\DecValTok{90}\NormalTok{, }\DecValTok{70}\NormalTok{, }\DecValTok{73}\NormalTok{)}
\NormalTok{kehadiran }\OtherTok{\textless{}{-}} \FunctionTok{c}\NormalTok{(}\DecValTok{85}\NormalTok{, }\DecValTok{70}\NormalTok{, }\DecValTok{85}\NormalTok{, }\DecValTok{80}\NormalTok{, }\DecValTok{60}\NormalTok{, }\DecValTok{80}\NormalTok{)}

\ControlFlowTok{for}\NormalTok{ (i }\ControlFlowTok{in} \DecValTok{1}\SpecialCharTok{:}\FunctionTok{length}\NormalTok{(nama)) \{}
  \ControlFlowTok{if}\NormalTok{ (nilai[i] }\SpecialCharTok{\textgreater{}=} \DecValTok{75} \SpecialCharTok{||}\NormalTok{ kehadiran[i] }\SpecialCharTok{\textgreater{}=} \DecValTok{80}\NormalTok{) \{}
\NormalTok{    status[i] }\OtherTok{\textless{}{-}} \StringTok{"Lulus"}
\NormalTok{  \} }\ControlFlowTok{else}\NormalTok{ \{}
\NormalTok{    status[i] }\OtherTok{\textless{}{-}} \StringTok{"Tidak Lulus"}
\NormalTok{  \}}
\NormalTok{\}}
\CommentTok{\#menampilkan hasil }
\NormalTok{hasil }\OtherTok{\textless{}{-}} \FunctionTok{data.frame}\NormalTok{(}\AttributeTok{Nama=}\NormalTok{nama, }\AttributeTok{Nilai=}\NormalTok{nilai, }\AttributeTok{Kehadiran=}\NormalTok{kehadiran, }\AttributeTok{Status=}\NormalTok{status)}
\FunctionTok{print}\NormalTok{(hasil) }
\end{Highlighting}
\end{Shaded}

\begin{verbatim}
##      Nama Nilai Kehadiran      Status
## 1  Stella    85        85       Lulus
## 2   Rafly    65        70 Tidak Lulus
## 3    Faiz    78        85       Lulus
## 4 Pingkan    90        80       Lulus
## 5   Zilda    70        60 Tidak Lulus
## 6   Kinan    73        80       Lulus
\end{verbatim}

\section{BAB 4 : Kesimpulan}\label{bab-4-kesimpulan}

\subsection{4.1 Komparansi antar fungsi}\label{komparansi-antar-fungsi}

\begin{itemize}
\item
  if vs ifelse :

  -\textgreater{} if digunakan untuk kondisi sederhana yang hanya
  terdapat data tunggal

  -\textgreater{} ifelse digunakan untuk kondisi yang lebih compleks
  dengan dengan memiliki data yang lebih dari 1 data atau pada vektor
  yang lebih besar.
\item
  for vs while :

  -\textgreater{} for loop digunakan ketika sudah mengetahui batas awal
  dan akhir pengulangan

  -\textgreater{} while loop digunakan ketika hanya tau syarat batas
  angka yang diberikan (misalnya ketika didalam soal batas angka harus
  dibawah atau sama dengan 30), tetapi pengulangannya diatur sendiri.
\item
  nested ifelse :

  -\textgreater{} nested if-else digunakan untuk menangani suatu kondisi
  yang bergantung pada kondisi sebelumnya. contoh pada study kasus
  diatas ingin memeriksa kategori grade nilai sebuah mahasiswa
  bergantung dengan kriteria yang sudah disebutkan disoal.
\item
  repeat loop
\end{itemize}

\begin{verbatim}
-\> repeat loop adalah pengulangan yang akan terus berjalan hingga diberikan perintah untuk berhenti. perbedaannya dengan loop lain adalah untuk menentukan pengulangan itu akan berhenti menggunakan perintah break. seperti pada study kasus diatas bahwa repeat loop diminta untuk melakukan pengulangan angka hingga jumlah atau totalnya mencapai 50. ketika total pencapaian melebihi 50 maka program akan berhenti dan itu menggunakan perintah break.
\end{verbatim}

\begin{itemize}
\item
  Logika dasar and (\&), or(\textbar)

  -\textgreater{} fungsi and (\&) digunakan untuk memastikan bahwa kedua
  kondisi yaitu (nilai ujian dan absensi) harus memenuhi kriteria agar
  mahasiswa lulus. jika salah satu kondisi tidak terpenuhi, siswa tidak
  akan lulus

  -\textgreater{} fungsi or (\textbar) digunakan untuk memastikan bahwa
  salah satu kondisi baik antara nilai ujian dan absensi telah memenuhi
  kriteria agar mahasiswa lulus. Jika salah satu kondisi tidak
  terpenuhi, siswa tidak akan lulus.
\end{itemize}

\section[DAFTAR PUSTAKA]{\texorpdfstring{DAFTAR
PUSTAKA\footnote{Muhammad Rizka Fadhli. November 2021. Intro to R Volume
  3 : Dasar-Dasar Bahasa Pemograman R

  Mohammad Reza Faisal. April 2016. Seri Belajar Pemrograman :
  Pengenalan Bahasa Pemograman R}}{DAFTAR PUSTAKA}}\label{daftar-pustaka1}

muh

\end{document}
